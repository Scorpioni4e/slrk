Slrk is a project I started to learn more about how Linux and x86\_64 works,
it is not intended to implement undetectable rootkit techniques but rather
explore the possibility we have to modify Linux behaviour.

Slrk is not a functional rootkit, but rather a library that gather techniques
and proofs of concept related to rootkit theory.

The main goal of this project is to find ways to hook the kernel by tricking
some parts of the kernel (mostly the interrupts procedures).
Some techniques presented in this document are experimental and may contain
a lot of bugs or races.
The implementation of the various methods dig into low-level aspects of
x86\_64 such as the IDT, GDT, debug registers, exceptions…
The interesting part of this project is that if you want to modify the
behaviour of a linux's sub-system, you first need to understand it.

The project also contains a test suites, which is a kernel module that use
the lib slrk to validate the new features and avoid regressions.
Coding such a module is also a good exercice to learn linux driver programming,
the tests suit uses common used API (module, debugfs, usermod…) to run the
various tests.

At last, a rootkit example is provided to illustrate how to use the lib to
implement rootkits. The example rootkit handle different common tasks such
as hiding code, sockets and files or even act as a keylogger.
