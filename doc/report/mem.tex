In many situations you want your module to rewrite some parts of the running
kernel, this include things like:
\begin{itemize}
  \item code: patching the code allow us to modify the behavior of a function;
  \item data: modifying the content of a variable, or even a pointer;
\end{itemize}
The base protection of the memory is permissions, this mechanisms provide a
set of per page flags that restrict the available operations on this page.
In order to implements certain techniques, we need to change data in
write-protected pages.

In order to understant the 3 presented methods, you have to know the basis of
memory management with a x86 processor \cite{kernel_memory}.

\subsection{Write Protect flag}
The first method is to directly use the feature of x86 CPU to disable the
write protection. The control registers can change the general behavior of the
CPU by setting or clearing a specific bit.
The CR0 register has various flags, but the one that is interesting in our case
is the \textit{Write Protect} flag.
\Quote{
Write Protect (bit 16 of CR0) - When set, inhibits supervisor-level procedures
from writing into read-only pages; when clear, allows supervisor-level
procedures to write into read-only pages (regardless of the  U/S bit setting;
see Section 4.1.3 and Section 4.6). This flag facilitates implementation of the
copy-on-write method of creating a new process (forking) used by operating
systems such as UNIX.}{IA-32 Architectures Software Developer’s Manuals, 2-14 Vol. 3A}

This mean that if we disable the write protection, the protection is disabled
globally and let use write anywhere in the memory without being annoyed by
the MPU.
The drawback is that we must ensure that we do not break kernel's features that
rely on it such as copy-on-write. \todo{Really used ?}
Before each memory modification that require us to change the WP flag, we will
disable interrupts and scheduling so write looks atomic on the CPU.
\cite{wp_safe}

\begin{lstlisting}[frame=single]
#define CR0_WP 0x00010000 // Write Protect Bit (CR0:16)

static inline unsigned long read_cr0(void) {
    unsigned long val;
    asm volatile("mov %%cr0,%0\n" : "=r" (val), "=m");
    return val;
}

static inline void write_cr0(unsigned long val) {
    asm volatile("mov %0,%%cr0\n" : : "r" (val), "m");
}

static void disable_wp(unsigned long addr) {
    local_irq_disable();
    barrier();
    write_cr0(read_cr0() & ~CR0_WP);
}

static void enable_wp(unsigned long addr) {
    write_cr0(read_cr0() | CR0_WP);
    barrier();
    local_irq_enable();
}
\end{lstlisting}

\subsection{Page Table Entry}
\todo{explain MMU}
Paging introduces the benefit of page-level protection.
Paging is achieved through the MMU, an unit that transform virtual addresses
into physical addresses based on the current page table.
The idea here is to directly modify the fields of page entry to change the
permissions.
A page table entry (PTE) is a 4-byte record that contain a serie of flags
used by the MMU.\\
\begin{figure}[h]
  \center
  \includegraphics[scale=0.5]{res/pte.png}
  \caption{Paging-Structure Entry}
\end{figure}

The RW flag in this structure enforce write protection across the page it
refers to. The idea is to simply flip this bit to 0 to disable the protection,
write the modifications in the page and flip it back.
Unlike the previous method, we change the permissions for a single page and
not globally.

\begin{lstlisting}[frame=single]
static void pte_set_rw(unsigned long addr) {
    pte_t *pte;
    unsigned int level;

    /* retrieve the page where the address reside */
    pte = lookup_address(addr, &level);
    /* Remove the write protection */
    if (pte->pte &~ _PAGE_RW)
        pte->pte |= _PAGE_RW;
}

static void pte_set_ro(unsigned long addr) {
    pte_t *pte;
    unsigned int level;

    /* retrieve the page where the address reside */
    pte = lookup_address(addr, &level);
    /* Set the page read-only */
    pte->pte = pte->pte &~_PAGE_RW;
}
\end{lstlisting}

\subsection{Shadow page mapping}
This technique was originally used by Ksplice\cite{ksplice} to write into
read-only regions inside the \cmd{stop\_machine} function that cannot use
\cmd{vmalloc/vfree}.
Memory is addressed through virtual mappings that point to a a specific
physical page.
The idea is to create an other view of the pages we want to write into.
Indeed, it is possible to  create two different virtual mapping that point to
the same physical paging. This trick let us have two different view of the same
physical page and thus drop the write protection).
This is possible thanks to the \cmd{vmap} function:
\SQuote{
vmap - map an array of pages into virtually contiguous space.
}
\vspace{-3mm}
\begin{lstlisting}[frame=single]
void *vmap (citestruct page ** pages,
            unsigned int count,
            unsigned long flags,
            pgprot_t prot);
\end{lstlisting}

This function return a contiguous space (where our read-only data reside) but
with our own permissions. We have created a different view of the datas, they
can be seen as read-only through their original page table entry, but
read-write through our mapping.

\begin{ccode}
static void *map_writable(void *addr, size_t len) {
    int i;
    void *vaddr;
    struct page **pages;
    int nr_pages = DIV_ROUND_UP(offset_in_page(addr) + len, PAGE_SIZE);
    void *page_addr = (void *)((unsigned long)addr & PAGE_MASK);

    /* Create and fill an array of the pages we want to map */
    pages = kmalloc(nr_pages * sizeof(*pages), GFP_KERNEL);
    if (pages == NULL)
        return NULL;
    for (i = 0; i < nr_pages; i++) {
        if (__module_address((unsigned long)page_addr) == NULL) {
            pages[i] = virt_to_page(page_addr);
            WARN_ON(!PageReserved(pages[i]));
        } else {
            pages[i] = vmalloc_to_page(page_addr);
        }
        if (pages[i] == NULL) {
            kfree(pages);
            return NULL;
        }
        page_addr += PAGE_SIZE;
    }

    /* Map the array of pages into virtually contiguous space with the 
     * PAGE_KERNEL permission (which is RW)
     */
    vaddr = vmap(pages, nr_pages, VM_MAP, PAGE_KERNEL);
    kfree(pages);
    if (vaddr == NULL)
        return NULL;
    return vaddr + offset_in_page(addr);
}
\end{ccode}

\cite{write_protected}
\todo{Shadow page mapping}
